%Pour compiler ce fichier avec bibtex :
%pdflatex 20140814_raport.tex
%bibtex 20140814_raport.aux
%pdflatex 20140814_raport.tex
%pdflatex 20140814_raport.tex

%lien dropbo : https://www.dropbox.com/s/c7n7242xsq36t4y/20140816_rapport.pdf


%%%%%%%%%%%%%%%%%%%%%%%%%%%%%%%%%%%%%%%%%%%
%        Packages utilisés                %
%%%%%%%%%%%%%%%%%%%%%%%%%%%%%%%%%%%%%%%%%%%



	%[twoside,12pt]

	%\documentclass[french,12pt]{report} %pour une police de 12 au cas où
	\documentclass[french]{report}

	\usepackage[T1]{fontenc}
	\usepackage[utf8]{inputenc} %Permet d'utiliser des caractères spéciaux : éèàçù etc. plus sûr que \usepackage[latin9]{inputenc}
	%\usepackage[latin9]{inputenc}

	\usepackage[french]{babel}

	\usepackage{geometry} %Personnaliser l'aspect géométrique du document.
	\geometry{top=3cm, bottom=3cm, left=3cm , right=3cm} %plus simple que le suivant, que je ne maîtrise pas...
	%\geometry{verbose,tmargin=2.5cm,bmargin=2.5cm,lmargin=1.5cm,rmargin=1.5cm,headheight=2.5cm,headsep=2.5cm,footskip=2.5cm}
	
	\usepackage{amstext} %??
	\usepackage{graphicx} %??
	%\usepackage{amsmath} %Utilisé plus bas
	\usepackage{amssymb} %??
	\usepackage{hyperref} %Celui-là il est surpuissant : il fait tous les liens hypertexte et intratexte. [hidelinks]==> pas de couleur 
	\hypersetup{colorlinks=false }
	%\hypersetup{hidelinks}


	\usepackage[footnotesize]{caption} %Moarf. Dans le doute je le garde.
	%\usepackage{graphicx}
	\usepackage{amsmath, amssymb} %oui mais ???
	%\usepackage[applemac]{inputenc}
	%\usepackage[T1]{fontenc}
	\usepackage{color} %???
	%\usepackage{times} %permet de choisir la police, mais la Latin Modern est tellement belle que je n'y touche pas.
	%\usepackage[francais]{babel} %déjà utilisé.
	%\usepackage{fullpage} %définit tout une mise en page qui bousille le package geometry !!!! ==> je ne l'utilise pas.
	\usepackage{natbib} %Pour la bibliographie ?
	\usepackage{colortbl} %pour les couleurs

	\usepackage[french]{minitoc} % papy : pour créer une table des matières juste avant l'annexe. Toujours pas réussi à utiliser ce truc infernal.
	\usepackage{tocloft}  
	\usepackage{titletoc}  
	\usepackage{appendix} %papy : utiliser la commande \appendix 
	\usepackage{eurosym} %papy : pour ajouter le symbole € en tapant \euro
	%\usepackage{minitoc}
	\usepackage{colortbl} %papy : pour les couleurs du tableau
	\usepackage{lipsum}
	\usepackage{amsthm} %théorèmes, etc.
	\usepackage{stmaryrd} %crochets d'entiers
	\usepackage{empheq} %équations encadrées
	\usepackage{mathrsfs}  %zoulies majuscules
	\usepackage{tikz} %graphiques en latex
	\usepackage{yhmath} %arc de cercle sur lettres : \wideparen{AB}
	\usepackage{bm} %\bm{truc} = truc en gras mais avec la bonne police

	%\usepackage{fancyhdr} %bas de pages %Plüthaar.
	%\pagestyle{fancy}


	\makeatletter

	%%%%%%%%%%%%%%%%quelques commandes dont je n'ai pas encore saisi l'utilité.
	\def \eps {\varepsilon}
	\def\percent{\%}
	\def\etal{{et al.}}



	%%%%%%%%%%%Commandes pour la bibliographie.
		\def\aj{\textit{The Astronomical Journal}}


		\def\nat{\textit{Nature}}
		\def\icarus{\textit{Icarus}}
		\def\aap{\textit{Astronomy and Astrophysics}}
		\def\grl{\textit{Geophysical Research Letters}}
		\def\jgr{\textit{Journal of Geophysical Research}}





	\newcommand{\figpath}{../figures/}
	\newcommand{\bibpath}{/Volumes/USER/robutel/Phil/Articles/base/}







	%%%%%%%%%%%%%%%%%%%%%%%%%%%%%% LyX specific LaTeX commands.%%% ==> et ???
	\DeclareRobustCommand{\greektext}{%
	  \fontencoding{LGR}\selectfont\def\encodingdefault{LGR}}
	\DeclareRobustCommand{\textgreek}[1]{\leavevmode{\greektext #1}}
	\DeclareFontEncoding{LGR}{}{}
	\DeclareTextSymbol{\~}{LGR}{126}
	\newcommand{\lyxmathsym}[1]{\ifmmode\begingroup\def\b@ld{bold}
	  \text{\ifx\math@version\b@ld\bfseries\fi#1}\endgroup\else#1\fi}


	\makeatother



	\newcommand{\HRule}{\rule{\linewidth}{0.5mm}}



	%%%%%%%%%%%%%%%%%%%%%%%%%%%%%%%%%%%%%%%%%%%%%%
	%Théorèmes, démonstrations, etc.
	%
	%%%%%%%%%%%%%%%%%%%%%%%%%%%%%%%%%%%%%%%%%%%%%

	\newtheorem{post}{Postulat}
	\newtheorem{theorem}{Théorème}
	\newtheorem{lemme}{Lemme}
	\newtheorem{definition}{Définition}
	\newtheorem{cor}{Corrolaire}
	\theoremstyle{plain}


	%%%%%%%%%%%%%%%%%%%%%%%%%%%%%%%%%%%%%%%%%%%%%
	%bas de page, haut de page
	%        etc.
	%		Plüthaar.
	%%%%%%%%%%%%%%%%%%%%%%%%%%%%%%%%%%%%%%%%%%%%%

	%\renewcommand{\headrulewidth}{1pt}
	%\fancyhead[C]{\textbf{page \thepage}} 
	%\fancyhead[L]{\leftmark}
	%\fancyhead[R]{machin}

	%\renewcommand{\footrulewidth}{1pt}
	%\fancyfoot[C]{Léo BERNUS} 
	%\fancyfoot[L]{truc}
	%\fancyfoot[R]{\leftmark}



\begin{document}

%%%%%%%%%%%%%%%%%%%%%%%%%%%%%%%%%%%%%%%%%%%%%%%%%%%%%%%%%%%%%%%%%%%%%%
%                                                                    %
%                          Page de garde                             %
%                                                                    %
%%%%%%%%%%%%%%%%%%%%%%%%%%%%%%%%%%%%%%%%%%%%%%%%%%%%%%%%%%%%%%%%%%%%%%




	\begin{titlepage}

	\begin{center}


	%%%%%%%%%%%%%
	%  Logos    %
	%%%%%%%%%%%%%



	%\begin{center}
	%  \includegraphics[width = 40mm]{logos/SANSELIPS.jpg} \hfill
	%  \includegraphics[width = 40mm]{logos/logo_IMCCE_small.jpg} \hfill
	%  \includegraphics[width = 40mm]{logos/UPMC_pantones_7504-166.jpg} \hfill
	%  \includegraphics[width = 35mm]{logos/ens.jpg}
	    
	%\end{center}





	%%%%%%%%%%%%%%%%%%%%%%%%%%%%%%%%%%%%%%%%%%%%%%%%%%%%%%%%%%%%%%%%%
	%   Code pour zoulie page de garde. NE PAS TOUCHER !!!!!!!!!    %
	%%%%%%%%%%%%%%%%%%%%%%%%%%%%%%%%%%%%%%%%%%%%%%%%%%%%%%%%%%%%%%%%%



	%%%%%%%%%%%%%%%%%
	%Si ce modèle de page de garde ne te plaît pas, cherches-en d'autres (sur internet, dans des manuels de LaTeX...). Les créer, c'est beaucoup plus compliqué.
	%%%%%%%%%%%%%%%%%





	\begin{minipage}[t]{0.3\textwidth}
	  \begin{flushleft}
	    
	    
	  \end{flushleft}
	\end{minipage}
	\begin{minipage}[t]{0.3\textwidth}
	  \begin{center}
	    
	  \end{center}
	\end{minipage}
	\begin{minipage}[t]{0.3\textwidth}
	  \begin{flushright}
	       
	  \end{flushright}
	\end{minipage}
	\\[1cm]


	\textsc{\Large }\\[0.6cm]
	\HRule \\[0.8cm]
	{\huge \bfseries Cahier des charges \\ L'Outre-Ça - version 3.0}\\[0.4cm]
	\HRule \\[1cm]

	\begin{minipage}[t]{0.3\textwidth}
	  \begin{flushleft} \large
	    \emph{Graphismes :}\\
	    Trollbernus
	  \end{flushleft}
	\end{minipage}
	\begin{minipage}[t]{0.6\textwidth}
	  \begin{flushright} \large
	    \emph{Programmation :} \\
		Dora  
		%Ça serait quand même pas mal d'obtenir leurs prénoms et de savoir d'où ils vennent...


	  \end{flushright}

	\end{minipage}
	%\\[2cm]




	%%%%%%%%%%%%%%%%%%%%%%%%%%%%%%%%%%%%%%%%
	%      Images pour la page de garde    %
	%%%%%%%%%%%%%%%%%%%%%%%%%%%%%%%%%%%%%%%%



	%%%Ça c'est pas pour moi mais je le garde on sait jamais..

	%\begin{minipage}[t]{0.6\textwidth}
	%  \begin{flushcenter} \large
	%    \emph{date de remise du rapport :} \\
	%    35 décembre 2048
	%  \end{flushcenter}	
	%\end{minipage}
	%\\[2cm]
	%



	%\begin{center}
	  %\includegraphics[scale=0.965]{mimas_noire_2.jpg} \\ %\hfill
	  %\includegraphics[width = 72mm]{mimas_rings.jpg} \hfill
	  %\includegraphics[width = 40mm]{UPMC_pantones_7504-166.jpg} \hfill
	  %\includegraphics[width = 35mm]{ens.jpg} 
	%\end{center}





	{\large   \today }






	\vfill



	\end{center}

	\end{titlepage}





	\pagebreak{}



%%%%%%%%%%%%%%%%%%%%%%%%%%%%%%%%%%%%%%%%%%%%%%%%%%%%%%%%%%%
%       troisième page : table des matières + abstract.    %
%%%%%%%%%%%%%%%%%%%%%%%%%%%%%%%%%%%%%%%%%%%%%%%%%%%%%%%%%%%



	\tableofcontents





	\pagebreak



\chapter{La page d'accueil}

	Telle qu'elle se présente actuellement : 
	\begin{enumerate}
		\item Le titre \emph{L'Outre-Ça} en calligraphie.
		\item En haut à gauche : un dessin-bouton qui mène au blog d'Oxylos.
		\item En haut à droite : idem pour Tarran.
		\item en bas à gauche : un dessin-bouton qui mène à la partie communautaire du site.
		\item en bas à droite : un dessin-bouton qui mène à la partie "à propos". 
	\end{enumerate}
	Les animations, les changements graphiques, etc., seront discutés en temps réel entre le graphiste et le webmaster.


\chapter{Site d'Oxylos}
	
	Se trouve à l'URL \url{www.outre-ca.fr/oxylos}

	Le site d'Oxylos se divise en deux principales parties : le blog et la galerie. Un accueil permet d'accéder à l'un ou à l'autre. Le header et le footer sont invariants dans tout le site d'Oxylos.

	\section{Header}
		Ce header est le même sur tout le site d'Oxylos.
		Le header regroupe les boutons suivants, de gauche à droite :
		\begin{enumerate}
			\item \emph{Accueil} : mène à la page d'accueil du site d'Oxylos (cf. section \ref{accueil_ox}).
			\item \emph{À propos} (cf. chapitre \ref{a_propos})
			\item \emph{Liens} : les sites que nous recommandons. Je préfère "liens" que "partenaires", soit dit en passant (cf. section \ref{liens}).
			\item \emph{Archives} : permet d'accéder aux archives du blog. (cf. section \ref{archives}).
			\item \emph{L'Outre-Ça} : image calligraphiée, lien mène à l'accueil de l'Outre-Ça.
			\item \emph{Communauté} : accède à la liste des inscrits (cf. section \ref{communaute}).
			\item \emph{Connexion} 
			\item \emph{Inscription} 
		\end{enumerate}

	\section{Footer}
		Ce footer est le même sur tout le site d'Oxylos.
		En haut à gauche : "flux RSS", lien vers le flux RSS du site d'Oxylos.
		En haut à droite : "L'Outre-Ça - 2013-2016 ; site réalisé par Dora et Oxylos Neest"
		En bas, centré : "Le contenu de ce site n'est pas libre de droit. Contacter les auteurs pour toute demande d'utilisation." \emph{Contacter les auteurs} est un bouton qui mène à la page \emph{contacts}.

	\section{Accueil}\label{accueil_ox}

		\emph{Attention !} Le header de l'accueil est graphiquement différent de celui de la galerie et du blog : il n'y a pas de flammes.

		La page d'accueil est particulièrement simple.
		On y trouve en dessous du header : la bannière, imposante et majestueuse.
		En dessous de la bannière, deux dessins-boutons :
		\begin{enumerate}
			\item À gauche : \emph{Blog}, qui mène au blog (cf. section \ref{blog_ox}).
			\item À droite : \emph{Galerie}, qui mène, devinez à quoi : à la galerie (cf. section \ref{galerie}). 
		\end{enumerate}

	\section{Blog} \label{blog_ox}
		\subsection{Point de vue du visiteur}
			On y trouve les articles rédigés par Oxylos. Ils sont classés du plus récent au plus ancien. Il y a un article par page.

			En dessous du footer, on trouove une bannière où il y a écrit "BLOG". Histoire que les lecteurs ne soient pas perdus. En dessous, un discret mais visible bouton "voir la galerie".

			Encore en dessous, nous trouvons l'article le plus récent. Tout le contenu de la liste suivante est inclus dans le cadre de l'article. Celui-ci contient, de haut en bas :
			\begin{enumerate}
			 	\item "Le [date] par Oxylos Neest".
			 	\item Titre de l'article.
			 	\item Corps de l'article.
			 	\item "$n$ commentaires" où $n\in \mathbb{N}$.
			\end{enumerate} 

			S'il s'agit de l'article le plus récent, en dessous de celui-ci nous trouvons simplement une flèche orientée vers la droite indiquant "Article suivant". S'il ne s'agit pas de l'article le plus récent, nous trouvons les deux flèches : "Article précédent" et "article suivant". 

			Losrque l'on clique sur "$n$ commentaires", on accède aux commentaires : la page s'ouvre juste en dessous de l'article et on commence directement à lire les commentaires.

		\subsection{Création d'un article}
			Pour Oxylos, la création des articles est presque identique à celle de la version 2.0 de l'Outre-Ça. Il y a cependant une fonctionnalité en plus : dans le corps du texte de l'article, Oxylos peut mettre soit des images ordinaires soit des super-images. Les images ordinaires s'uploadent exactement comme dans la version 2.0 de l'Outre-Ça. En revanche, quand Oxylos veut mettre une super-image, il doit en choisir une disponible dans la galerie (cf. sections \ref{galerie} et \ref{lecteur_images}).

	\section{Galerie} \label{galerie}
		\subsection{Point de vue du visiteur}
			On y trouve toutes les créations graphiques d'Oxylos lisible avec le super-lecteur d'images (cf. chapitre \ref{lecteur_images}). 

			Tout en haut, une bannière "galerie". En dessous, un discret mais visible bouton "Voir le blog".

			De gauche à droite, une série de boutons :
			\begin{enumerate}
				\item \emph{Tout} : donne accès à tous les éléments de la galerie.
				\item \emph{Bandes dessinées} : donne accès aux BDs de la galerie.
				\item \emph{Illustrations} : donne accès aux illustrations de la galerie.
				\item \emph{Croquis} : donne accès aux croquis de la galerie. 
			\end{enumerate}

			En dessous, les éléments clicables de la galeries qui ouvrent le lecteur, classés du plus récent au plus ancien.
			Le contenu de chaque élément est discuté au chapitre \ref{lecteur_images}.

		\subsection{Point de vue de l'auteur}
			Un bouton \emph{modifier/supprimer} à côté de chaque super-image. Quand l'auteur n'est pas dans la partie \emph{Tout}, au dessus des éléments de la galerie se trouve un bouton \emph{Ajouter une super-image}.


	\section{Archives}\label{archives}
		Contient la liste des articles d'Oxylos.

		Tout en haut, une bannière où il y a écrit "Archives".

		Ensuite, dans le même cadre que les articles d'Oxylos, on trouve quelque chose rangé de la façon suivante :

		Un bouton "Retour au blog".

		\begin{itemize}
			\item 2016
			\begin{itemize}
				\item Janvier
				\begin{itemize}
					\item Bienvenue sur le site ([date])
					\item De l'arbre sur la montagne ([date])
				\end{itemize}
				\item Février
				\begin{itemize}
					\item Festival d'Angoulême ([date])
					\item etc.
				\end{itemize}
			\end{itemize}
			\item 2017
			\begin{itemize}
				\item etc.
			\end{itemize}
		\end{itemize}
		Chaque élément étant déroulant : on peut replier ou déplier "2016" ou juste "janvier" etc. si on ne veut pas voir ce qu'il y a.

		En dessous de tout ceci, deux boutons : \emph{retour au blog} et \emph{Voir la galerie}.

\chapter{Site de Tarran}
	\emph{Remarque importante :} Tarran m'a dit récemment qu'il songeait à changer de pseudo pour écrire.

	Tout comme Oxylos, le site de Tarran a sa propre URL : quelque chose comme \url{www.outre-ca.fr/tarran}. La structure du site de Tarran est plus simple que celle du site d'Oxylos, mais aussi plus originale, car elle se présente sous la forme d'un web-roman.

	\section{Header}
		C'est le même sur tout le site de Tarran.
		Le header regroupe les boutons suivants, de gauche à droite :
		\begin{enumerate}
			\item \emph{Accueil} : mène à la première page  du site de Tarran.
			\item \emph{À propos} (cf. chapitre \ref{a_propos})
			\item \emph{Liens} : les sites que nous recommandons. Je préfère "liens" que "partenaires", soit dit en passant (cf. chapitre \ref{liens}).
			%\item \emph{Archives} : permet d'accéder aux archives du blog. (cf. section \ref{archives}).
			\item \emph{L'Outre-Ça} : image calligraphiée, lien mène à l'accueil de l'Outre-Ça.
			\item \emph{Communauté} : accède à la liste des inscrits (cf. section \ref{communaute}).
			\item \emph{Connexion} 
			\item \emph{Inscription} 
		\end{enumerate}

	\section{Footer}
		C'est le même sur tout le site de Tarran. C'est exactement le même que celui d'Oxylos, sauf que le flux RSS renvoie au flux de Tarran.

	\section{Structure du web-roman}
		Il y a trois articles par page. Chaque groupe de trois articles représente un chapitre. Les chapitres sont regroupés en tomes, et à long terme les tomes seront peut-être regroupés en parties.
		Quand les lecteurs arrivent sur le site de Tarran, ils tombent sur le premier chapitre du tome un de la première partie. 

		\subsection{Structure d'un chapitre}
			Dans un chapitre, nous trouvons, de haut en bas :

			Deux boutons : \emph{Table des matières} (cf. section \ref{toc}), et \emph{Reprendre depuis la dernière lecture}. Il faut se connecter pour accéder à cette fonctionnalité. En effet, le site peut mémoriser, pour chaque compte, la dernière page du site de Tarran qu'il a visité (hors table des matières bien sûr), et l'y renvoyer immédiatement. Si l'utilisateur n'est pas connecté et clique sur ce bouton, une fenêtre s'ouvre et indique : "il faut se connecter pour accéder à cette fonctionnalité du site" avec bien sûr les boutons \emph{Connexion} et \emph{Incription}. 

			Ensuite, nous trouvons, dans le même cadre que les articles, d'abord le titre de la partie, puis le titre du tome, puis le titre du chapitre qui contient les trois articles. Les titres de la partie et du tome sont écrits en petit et le titre du chapitre en gros. Quelque chose comme cela :

			\phantom{lulz}

			Partie 1 : végétaux humains. - Tome 1 : De l'arbre sur la montagne.

			\phantom{lulz}

			\begin{huge}\begin{bfseries} Chapitre 1 : Enracinement. \end{bfseries} \end{huge}

			\phantom{lulz}

			Plus bas, les trois articles, dans trois cadres, dont le contenu est similaire à celui d'Oxylos (cf. section \ref{blog_ox}).

			Et tout en bas : 
			\begin{itemize}
				\item S'il s'agit du tout premier chapitre, on trouve deux boutons :
				\begin{itemize}
					\item \emph{Chapitre suivant} mène à la page suivante contenant trois articles.
					\item \emph{Tome suivant} mène au début du tome suivant (lorsqu'il existera, ce qui ne sera pas le cas lors des premiers mois du site).
				\end{itemize}
				\item S'il s'agit d'un chapitre du premier tome, on trouve trois boutons :
				\begin{itemize}
					\item \emph{Chapitre précédent} mène à la page précédente contenant trois articles.
					\item \emph{Chapitre suivant} mène à la page suivante contenant trois articles.
					\item \emph{Tome suivant} mène au début du tome suivant (lorsqu'il existera, ce qui ne sera pas le cas lors des premiers mois du site).
				\end{itemize}
				\item S'il s'agit dun chapitre quelconque, on trouve quatre boutons :
				\begin{itemize}
					\item \emph{Tome précédent} mène au début du tome précédent (quand il existera).
					\item \emph{Chapitre précédent} mène à la page précédente contenant trois articles.
					\item \emph{Chapitre suivant} mène à la page suivante contenant trois articles.
					\item \emph{Tome suivant} mène au début du tome suivant (quand il existera). 
				\end{itemize}
				\item S'il s'agit d'un chapitre du dernier tome, on trouve trois boutons :
				\begin{itemize}
					\item \emph{Tome précédent} mène au début du tome précédent (quand il existera).
					\item \emph{Chapitre précédent} mène à la page précédente contenant trois articles.
					\item \emph{Chapitre suivant} mène à la page suivante contenant trois articles.
				\end{itemize}
				\item S'il s'agit du dernier chapitre, on trouve deux boutons :
				\begin{itemize}
					\item \emph{Tome précédent} mène au début du tome précédent (quand il existera).
					\item \emph{Chapitre précédent} mène au chapitre précédent.
				\end{itemize}
			\end{itemize}

			Tout cela semble compliqué mais quand on y pense c'est logique.

		\subsection{Table des matières} \label{toc}

			Dans un cadre semblable à celui des articles, nous trouvons ce qui suit, de haut en bas.
			Un bouton \emph{Reprendre depuis la dernière lecture}.

			Quelque chose qui a la structure suivante :
			\begin{itemize}
				\item Partie 1
				\begin{itemize}
					\item Tome 1 : [titre]
					\begin{itemize}
						\item Chapitre 1 : [Titre]
						\begin{itemize}
							\item De l'arbre sur la montagne [date]
							\item Lorem ipsum [date]
							\item La terre et le ciel [date]
						\end{itemize}
						\item Chapitre 2 : [Titre]
						\begin{itemize}
							\item De l'arbre sur la montagne [date]
							\item Lorem ipsum [date]
							\item La terre et le ciel [date]
						\end{itemize}
					\end{itemize}
					\item Tome 2 : [Titre]
					\begin{itemize}
						\item Chapitre 1 : [Titre]
						\begin{itemize}
							\item De l'arbre sur la montagne [date]
							\item Lorem ipsum [date]
							\item La terre et le ciel [date]
						\end{itemize}
						\item Chapitre 2 : [Titre]
						\begin{itemize}
							\item De l'arbre sur la montagne [date]
							\item Lorem ipsum [date]
							\item La terre et le ciel [date]
						\end{itemize}
					\end{itemize}
				\end{itemize}
				\item etc.
			\end{itemize}

			J'ai mis deux chapitres par tome mais le nombre est totalement arbitraire et dépendra du bon vouloir de notre Tarran international.

		\subsection{Côté auteur}
			\subsubsection{Présentation générale}

				Tarran aura des possibilités particulières. Quand il cliquera sur "Tarran" (en haut à droite du header), il aura accès aux boutons suivants :
				\begin{itemize}
					\item \emph{Mon compte}
					\item \emph{Mon \oe{}uvre} mène à la table des matières.
					\item \emph{Mode lecture} si le mode auteur est actif, \emph{Mode auteur} si le mode lecture est actif (cf. paragraphe \ref{auteurs})
					\item \emph{Déconnexion}
				\end{itemize}
				Comme la structure des chapitres est particulièrement compliquée, la création se fera à partir de la table des matières (possible uniquement quand il est connecté bien sûr). Dans ce cas, il voit exactement la même table des matières que les lecteurs mais avec des boutons en plus que je détaille ci-dessous (les boutons sont en \begin{bfseries} gras \end{bfseries}) :

				Quelque chose qui a la structure suivante :
				\begin{itemize}
					\item Partie 1 \begin{bfseries} supprimer/modifier \end{bfseries}
					\begin{itemize}
						\item Tome 1 : [titre] \begin{bfseries} supprimer/modifier \end{bfseries}
						\begin{itemize} 
							\item Chapitre 1 : [Titre] \begin{bfseries} supprimer/modifier \end{bfseries}
							\begin{itemize}
								\item De l'arbre sur la montagne [date] \begin{bfseries} supprimer/modifier \end{bfseries}
								\item Lorem ipsum [date] \begin{bfseries} supprimer/modifier \end{bfseries}
								\item La terre et le ciel [date] \begin{bfseries} supprimer/modifier \end{bfseries}
							\end{itemize}
							\item Chapitre 2 : [Titre] \begin{bfseries} supprimer/modifier \end{bfseries}
							\begin{itemize}
								\item De l'arbre sur la montagne [date] \begin{bfseries} supprimer/modifier \end{bfseries}
								\item Lorem ipsum [date] \begin{bfseries} supprimer/modifier \end{bfseries}
								\item La terre et le ciel [date] \begin{bfseries} supprimer/modifier \end{bfseries}
							\end{itemize}
						\end{itemize}
						\item Tome 2 : [Titre] \begin{bfseries} supprimer/modifier \end{bfseries}
						\begin{itemize}
							\item Chapitre 1 : [Titre] \begin{bfseries} supprimer/modifier \end{bfseries}
							\begin{itemize}
								\item De l'arbre sur la montagne [date] \begin{bfseries} supprimer/modifier \end{bfseries}
								\item Lorem ipsum [date] \begin{bfseries} supprimer/modifier \end{bfseries}
								\item La terre et le ciel [date] \begin{bfseries} supprimer/modifier \end{bfseries}
							\end{itemize}
							\item Chapitre 2 : [Titre] \begin{bfseries} supprimer/modifier \end{bfseries}
							\begin{itemize}
								\item De l'arbre sur la montagne [date] \begin{bfseries} supprimer/modifier \end{bfseries}
								\item \begin{bfseries} + Ajouter un article \end{bfseries} 
							\end{itemize}
							\item \begin{bfseries} + Ajouter un chapitre \end{bfseries}
						\end{itemize}
						\item \begin{bfseries} + Ajouter un tome \end{bfseries}
					\end{itemize}
					\item \begin{bfseries} + Ajouter une partie \end{bfseries}
				\end{itemize}

			\phantom{troud}

			Remarque : si le dernier chapitre contient déjà trois articles, il n'y aura pas de bouton \emph{Ajouter un article}.

			Lors de l'ouverture du site, cette page aura l'allure suivante :

			\begin{bfseries} + Ajouter une partie \end{bfseries}

			\subsubsection{Description détaillée de chaque bouton}

				\emph{Ajouter un article} est exactement la même chose que pour Oxylos.

				\emph{Ajouter un chapitre} ouvre un formulaire où Tarran devra écrire :
				\begin{itemize}
					\item Le titre du chapitre.
					\item Le titre du premier article de ce chapitre.
					\item Le corps du texte de l'article.
				\end{itemize}

				\emph{Ajouter un tome}
				ouvre un formulaire où Tarran devra écrire :
				\begin{itemize}
					\item Le titre du tome.
					\item Le titre du premier chapitre du tome.
					\item Le titre du premier article de ce chapitre.
					\item Le corps du texte de l'article.
				\end{itemize}

				\emph{Ajouter un tome} ouvre un formulaire où Tarran devra écrire :
				\begin{itemize}
					\item Le titre de la partie.
					\item Le titre du premier tome de la partie.
					\item Le titre du premier chapitre de ce tome.
					\item Le corps du premier article de ce chapitre.
					\item Le corps du texte de ce chapitre.
				\end{itemize}

				\emph{Supprimer} : ai-je vraiment besoin de préciser ?
				
				\emph{Modifier} :
				\begin{itemize}
					\item S'il s'agit d'un chapitre, d'un tome ou d'une partie, cela permet d'en modifier le titre.
					\item S'il s'agit d'un article, ce bouton permet d'en modifier le titre et le corps du texte.
				\end{itemize}

				\phantom{gras}

				Cette façon de faire lui laissera plus de souplesse dans l'execution de la création.













\chapter{Lecteur de BD/images} \label{lecteur_images}
	Le lecteur de BD/d'images est un programme qui a en entrée un ou plusieurs fichiers d'images. Quand on l'active, tout le reste du site s'assombrit, et la première image apparaît. Remarque : seules les suites d'images de la galerie peuvent apparaître dans le lecteur de BD/d'images.

	\section{Côté utilisateur}
		
		Une super-image, avant qu'on ne clique dessus, se présente de la façon suivante :
		
		Dans un cadre particulier identifiable, nous trouvons une petite image de présentation, un titre, et une date de publication. Si l'on est dans la galerie, on trouve également une brève description. Si on est dans un article, la description n'apparaît pas car risque d'être redondante avec le texte de l'article.

		Voici les fonctions que je veux voir sur ce lecteur d'images :
		\begin{enumerate}
			\item Un zoom matérialisé par de discrets boutons + et - en bas de l'écran, mais qui peut aussi s'activer avec la molette de la souris, et aussi par les touches + et - du clavier. 
			\item Déplacement au sein de l'image quand on a zoomé : le clic de la souris qui permet en quelque sorte "d'attraper" l'image. Il y a également des curseurs de défilement complètement à droite et complètement en bas de l'écran. Ces curseurs doivent être aussi discrets que possible, par exemple en ayant une couleur sombre tout comme le fond de l'image.
			\item Remarque : le zoom et le déplacement peuvent être implémentés en même temps pour les écrans tactiles ou les trackpads des mac : on peut zoomer et se déplacer en touchant l'écran avec deux doigts.
			\item Passage à l'image suivante, matérialisé par deux boutons à gauche et à droite de l'écran, mais aussi par les touches des flèches gauche et droite du clavier.
			\item Quitter le lecteur d'images : un bouton en haut à gauche ou en haut à droite. Possible aussi en appuyant sur échappe ou encore en cliquant dans le noir, ailleurs que sur l'image.
			\item Possibilité de passer en mode plein écran.
		\end{enumerate}
		Remarque : quand la personne qui lit zoome beaucoup (et si l'image en question est assez grande pour le permettre), l'image remplit tout l'espace visuel du navigateur. Cela permet de donner une expérience de lecture inédite. 


	\section{Côté auteur}

		\subsection{Organisation générale}
			Les super-images se créent à partir de la galerie. Quand Oxylos est connecté et visite la galerie, juste au dessus des objets de la galerie il y a un bouton "ajouter une super-image". Attention, ce bouton n'est pas accessible quand Oxylos est dans la partie "tout". Par exemple, s'il est dans la partie "BD" et qu'il ajoute une super-image, cette super-image sera automatiquement rangée dans la partie "BD" de la galerie. Cf. section \ref{galerie} pour mieux comprendre l'organisation de la galerie.

		\subsection{Détails de la création d'une super-image}

			Quand Oxylos crée une super-image, il arrive sur le formulaire qui contient les champs à remplir suivants.
			\begin{itemize}
				\item Titre de la super-image.
				\item Breve description de la super-image, avec éventuellement un lien vers un article qui en parle.
				\item Un bouton \emph{Image de présentation} : petite image visible dans la galerie ou sur un article du blog.
				\item Un bouton \emph{Ajouter une image}.
				\item Un bouton \emph{Aperçu}.
				\item Un bouton \emph{Ajouter/publier} bref tu vois l'délire, c'est comme dans les articles.
			\end{itemize}
			\phantom{troud}

			Détaillons quelques boutons. 

			\emph{Ajouter une image} se fait en deux étapes. D'abord, il faut uploader l'image en question (ou la sélectionner parmi les images hébergées si l'ajout d'images se fait directement en ssh). Ensuite, il faut décider des paramètres d'affichage de l'image. \emph{Léo discutera de cela avec Laurent. Pour le moment, le lecteur d'image tel qu'implémenté fera l'affaire.}

			\emph{Aperçu} permet à Oxylos de visualiser sa création avant de la publier.






\chapter{Communauté} \label{communaute}
	Une partie importante qui mérite d'être plus développée que dans les versions précédentes des comptes. 
	\section{Les grades des comptes}
		Dans cette section, j'énumère par ordre décroissant d'importance les différents grades des comptes. Chaque grade de classe $n$ peut faire monter en grade tous les comptes de grade inférieur à $n$.

		\subsection{L'administrateur}
			L'administrateur est un compte unique. C'est le big boss, le patron, autrement dit, Guillaume. Comme c'est lui qui code tout le site, techniquement il peut faire absolument tout ce qu'il désire. C'est le dieu dans son site.

		\subsection{Auteurs} \label{auteurs}
			Les auteurs du site sont des demi-dieux. Ils seront au nombre de deux : Oxylos et Tarran. Les auteurs peuvent faire énormément de choses. Voici l'énumération de leur possibilités :
			\begin{enumerate}
				\item Créer, écrire, modifier et supprimer des articles de blog.
				\item Modérer les commentaires, les comptes, bref, tout ce qui s'occupe de gérer la communauté des inscrits.
				\item Écrire des commentaires, modifier ou supprimer tous les commentaires écrits.
				\item Changer les bannières et les fonds des différentes pages du site.
				\item Changer les polices d'écriture des titres etc. au sein de tout le site. 
				\item Envoyer des mails aux inscrit ayant renseigné leur adresse mail. Donc, voire les mails des inscrits.
				\item Envoyer des messages aux inscrits, sur la boîte de messagerie interne du site.
				\item Utiliser le logiciel de messagerie instantanée.
			\end{enumerate}
			Cette grande souplesse me permettra, si mon style graphique évolue trop, de pouvoir continuer à utiliser ce site sans que l'administrateur n'ait de travail supplémentaire à faire.

			Les auteurs ont deux modes de lecture du site : le mode lecture et le mode auteur. Cela se matérialise par un bouton du menu déroulant en haut à droite du header :
			\begin{itemize}
				\item \emph{Mode lecture} si l'auteur est en mode auteur.
				\item \emph{Mode auteur} si l'auteur est en mode lecture.
			\end{itemize}
			Cela permet aux auteurs de profiter du design original du site sans se taper les boutons "modifier/supprimer" partout. Quand un auteur se connecte, le mode est celui qu'il avait lors de sa dernière connexion. Lors de la toute première connexion d'un auteur, le mode par défaut est le mode auteur.

		\subsection{Comptes d'élite}
			Les comptes d'élite représentent, la plupart du temps, les amis des auteurs et de l'administrateur. Cela permet simplement à quelques personnes sélectionnées d'utiliser le logiciel de discussion instantannée pour ne pas avoir à subir les horreurs absolues que sont skype, fassebouque, etc., ni sans se faire espionner. Voici ce que peuvent faire les comptes d'élite :
			\begin{enumerate}
				\item Écrire des commentaires, modifier ou supprimer tous les commentaires écrits. Ainsi, les comptes d'élite peuvent aider les auteurs dans la modération. 
				\item Envoyer des mails aux inscrit ayant renseigné leur adresse mail. Donc, voire les mails des inscrits.
				\item Envoyer des messages des inscrits, sur la boîte de messagerie interne du site.
				\item Utiliser le logiciel de messagerie instantanée.
			\end{enumerate}
			Les comptes d'élite sont donc forcément des personnes de confiance.


		\subsection{Le compte de base}
			Le compte que tout le monde peut avoir en indiquant un pseudo et en choisissant un mot de passe. En voici les possibilités :
			\begin{enumerate}
				\item Écrire des commentaires. Modifier ou supprimer ses propres commentaires.
				\item Envoyer des messages des inscrits, sur la boîte de messagerie interne du site. 
			\end{enumerate}

		Tous les comptes peuvent écrire et modifier leur page de profil.


	\section{Visualisation des comptes}

		\subsection{Header}
			Le header est le même sur toute la partie communauté. Il contient les boutons suivants, de gauche à droite :
			\begin{itemize}
				\item \emph{Accueil} : mène à la page d'accueil du site d'Oxylos (cf. section \ref{accueil_ox}).
				\item \emph{À propos} (cf. chapitre \ref{a_propos})
				\item \emph{Liens} : les sites que nous recommandons. Je préfère "liens" que "partenaires", soit dit en passant (cf. section \ref{liens}).
				\item \emph{Oxylos} mène au site d'Oxylos.
				\item \emph{Tarran} mène au site de Tarran.
				\item \emph{L'Outre-Ça} bouton calligraphié, mène à la page d'accueil du site.
				\item \emph{Connexion}.
				\item \emph{Inscription}.
			\end{itemize}

		\subsection{Footer}
			Le footer est le même sur toute la partie communauté. Il contient :

			En haut à droite : "L'Outre-Ça - 2013-2016 ; site réalisé par Dora et Oxylos Neest"
			En bas, centré : "Le contenu de ce site n'est pas libre de droit. Contacter les auteurs pour toute demande d'utilisation." \emph{Contacter les auteurs} est un bouton qui mène à la page \emph{contacts}.



		\subsection{Accueil de la communauté} \label{accueil_comm}
			Quand on clique sur "communauté" du header ou sur le bouton en bas à gauche de la page d'accueil, on accède à la liste des profils. Il y a une liste des comptes des inscrits ; on peut voir le profil de chacun en cliquant sur son pseudo. Les comptes sont séparés par grade. Tout en haut, il y a l'administrateur. En dessous, les auteurs. Encore en dessous, les comptes d'élite. Et tout en bas, les comptes ordinaires.

			Cette liste se présente donc de la façon suivante (à l'intérieur d'un cadre) :
			\begin{itemize}
				\item Administrateur : Dora.
				\item Auteurs :
				\begin{itemize}
					\item Oxylos
					\item Tarran
				\end{itemize}
				\item Modérateurs :
				\begin{itemize}
					\item IcarosWilos
					\item Sötnos Pepper
					\item Kodmowara
				\end{itemize}
				\item Comptes ordinaires :
				\begin{itemize}
					\item Jean-Jacques
					\item Bernard
					\item Jean-Roger
					\item Jean-Kévin
					\item Jean Dugros
					\item Coq
					\item Coq
					\item Coq
					\item Coq
					\item Coq
					\item Coq
					\item Coq
					\item Coq
					\item Coq
					\item Coq
					\item Coq
					\item Coq
					\item Coq
					\item Coq
					\item Coq
					\item Coq
					\item Coq
					\item Coq
					\item Coq
					\item Coq
					\item Coq
					\item Coq
					\item Coq
					\item Coq
					\item Coq
					\item Coq
					\item Coq
					\item Coq
					\item Coq
					\item Coq
					\item Coq
					\item Coq
					\item Coq
				\end{itemize}
			\end{itemize}

		\subsection{Accès aux pages de profil}
			Il y a deux façons et demi d'accéder à la page de profil d'un inscrit. La première façon, c'est en passant par le bouton du header "voir les inscrits" et de cliquer sur le profil. La deuxième façon, c'est qu'à chaque fois qu'un inscrit fait quelque chose (commentaire, article, super-suite d'immage lue par le super-lecteur, etc.), son nom est écrit à côté de ce qu'il fait. Il suffit alors de cliquer dessus pour voir son profil. La demi façon restante est que, lorsque vous êtes connecté, dans le header, à la place du bouton "se connecter/s'inscrire", il y a un bouton qui porte le nom du pseudo du connecté. En cliquant dessus, la personne connectée accède à sa page de profil qu'il peut modifier.

		\subsection{Page de profil vue de l'extérieur ou par un compte ordinaire}
			\begin{enumerate}
				\item Pseudo.
				\item Image de profil, ou avatar.
				\item Bouton pour le contacter sur sa messagerie interne du site.
				\item Un unique article type article de blog où l'inscrit parle de lui.
				\item Historique de son activités : commentaires écrits, articles écrits, super-suite d'images (lisibles par le super-lecteur) créées. En cliquant dessus on arrive directement sur l'objet en question.
				\item Un bouton "retour à la liste des profils".
			\end{enumerate}

		\subsection{Profil vu par soi-même lorsque l'on est connecté}
			\begin{enumerate}
				\item Tout ce qu'il y a dans la section précédente, mais avec un petit bouton "modifier" en plus s'il y a déjà du contenu, "ajouter" sinon.
				\item Un bouton pour accéder à ses messages sur la messagerie interne du site.
				\item Possibilité de modifier, en quelque sorte, "l'article" où l'inscrit parle de lui : il peut en profiter pour mettre des liens vers ses différents sites, ainsi que dire quelques mots sur lui. En revanche, impossible de créer une super-suite d'images visible dans la galerie.
				\item Un bouton pour accéder au logiciel de messagerie instantannée. Disponible que pour les modérateurs et classes supérieures.
			\end{enumerate}

		\subsection{Profil vu par les modérateurs, les auteurs ou l'administrateur}
			\begin{enumerate}
				\item Tout ce qu'il y a dans la section précédente. Les modérateurs peuvent vraiment tout faire, y compris regarder la messagerie interne au site des membres ! Mais ils utilisent leur pouvoir avec parcimonie.
				\item Un bouton "supprimer le profil". Valable uniquement sur les profils de rang inférieurs à soi.
				\item Bouton "promotion" : permet de faire monter en grade un profil. Les règles sont les suivantes : l'administrateur a tous les droits, les auteurs nomment les modérateurs. Mais ils peuvent aussi retirer un modérateur (ou un auteur pour l'admin) de ses fonctions. 
			\end{enumerate}

		\subsection{Messagerie}
			Il me semble que la messagerie de l'ancien site n'était pas si dégueulasse que ça. On pourrait simplement ajouter quelques fonctionnalités : un bouton "répondre" pour les comptes ordinaires, un bouton "répondre par mail" (si l'utilisateur a renseigné son mail) pour les modérateurs et plus. 


\chapter{Liens} \label{liens}
	
	Le header, le footer et le design sont ceux de la partie communauté (cf. chapitre \ref{communaute})

	Ce que tu avais fait dans l'ancien Outre-Ça était presque parfait. Seul le design sera modifié.

\chapter{À propos} \label{a_propos}

	À rédiger de nouveau. Mentionner l'ancien site, donner la possibilité de le visiter.


\chapter{Logiciel de dialogue instantané} \label{dialogue}
	\emph{À rédiger en collaboration avec Laurent\ldots}

\chapter{Archive ancestrale : L'Outre-Ça 2.0}
	On hébergera toutes les vieilles publications de l'ancien site sur notre serveur. On pourra y accéder via le \emph{À propos}. Par exemple dans le répertoire \url{outre-ca.fr/archives_ancestrales}

	Il faudrait faire en sorte que l'url \url{www.perso.crans.org/gmastio} ainsi que toutes les branches renvoient directement à \url{www.outre-ca.fr}.


\chapter{Perspective pour la version 3.1}
	
	\begin{itemize}
		\item Développer la partie communautaire et créer un forum. 
		\item Développer plein de petits trucs utiles quand on possède un serveur pour tous les comptes de rang supérieurs ou égaux aux modérateurs.
		\item \emph{D'autres idéees s'ajouteront.}
	\end{itemize}


\end{document}


